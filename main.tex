\documentclass[10pt,a4paper,final,oneside,onecolumn,notitlepage]{article}
\usepackage[utf8]{inputenc}
\usepackage[spanish]{babel}
\usepackage{amsmath}
\usepackage{amsfonts}
\usepackage{amssymb}
\usepackage{graphicx}
\usepackage[left=2cm,right=2cm,top=2cm,bottom=2cm]{geometry}
\author{Lucio Flores}
\title{Taller de \LaTeX }
%\date{ola k ase}

\begin{document}
\maketitle
\abstract{Como parte de las actividades del Festival Latinoaméricano de Instalación de Software Libre 2013, el IPN a tráves de la ESIME Zacatenco, la ESCOM, la UPIITA y la UPIICSA se complace en ofrecer este pequeño pero útil taller de \LaTeX en el cual veremos las ventajas de usar este procesador de texto avanzado así como algunas desventajas. Nos centraremos en las cualidades en las que \LaTeX es superior ampliamente a Microsoft Word y Libre Office.}


\section{¿Qu\'e es \LaTeX?}{
\LaTeX es un sistema de composición de textos, orientado especialmente a la creación de libros, documentos científicos y técnicos que contengan fórmulas matemáticas. \LaTeX está formado por un gran conjunto de macros de \TeX, escrito por Leslie Lamport en 1984, con la intención de facilitar el uso del lenguaje de composición tipográfica, , creado por Donald Knuth. Es muy utilizado para la composición de artículos académicos, tesis y libros técnicos, dado que la calidad tipográfica de los documentos realizados con LaTeX es comparable a la de una editorial científica de primera línea. \LaTeX es software libre bajo licencia LPPL.
}

\section{¿Por qu\'e (no) usar \LaTeX?}
\section{Tipos de Documento}
\section{Ejemplos variados}

\end{document}
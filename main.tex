\documentclass[10pt,a4paper,final,oneside,onecolumn,notitlepage]{article}
\usepackage[utf8]{inputenc}
\usepackage[spanish]{babel}
\usepackage{amsmath}
\usepackage{amsfonts}
\usepackage{amssymb}
\usepackage{graphicx}
\usepackage{url}
\usepackage[left=2cm,right=2cm,top=2cm,bottom=2cm]{geometry}
\author{Lucio Flores}
\title{Taller de \LaTeX }
%\date{ola k ase}

\begin{document}
\maketitle
\abstract{Como parte de las actividades del Festival Latinoaméricano de Instalación de Software Libre 2013, el IPN a tráves de la ESIME Zacatenco, la ESCOM, la UPIITA y la UPIICSA se complace en ofrecer este pequeño pero útil taller de \LaTeX en el cual veremos las ventajas de usar este procesador de texto avanzado así como algunas desventajas. Nos centraremos en las cualidades en las que \LaTeX es superior ampliamente a Microsoft Word y Libre Office.}


\section{¿Qu\'e es \LaTeX?}{
\LaTeX es un sistema de composición de textos, orientado especialmente a la creación de libros, documentos científicos y técnicos que contengan fórmulas matemáticas. \LaTeX está formado por un gran conjunto de macros de \TeX, escrito por Leslie Lamport en 1984, con la intención de facilitar el uso del lenguaje de composición tipográfica, , creado por Donald Knuth. Es muy utilizado para la composición de artículos académicos, tesis y libros técnicos, dado que la calidad tipográfica de los documentos realizados con LaTeX es comparable a la de una editorial científica de primera línea. \LaTeX es software libre bajo licencia LPPL.

\LaTeX es un sistema de composición de textos que está formado mayoritariamente por órdenes construidas a partir de comandos de \TeX , un lenguaje <<de nivel bajo>>, en el sentido de que sus acciones últimas son muy elementales pero con la ventaja añadida de «poder aumentar las capacidades de \LaTeX utilizando comandos propios del \TeX descritos en The TeXbook».3 4 Esto es lo que convierte a LaTeX en una herramienta práctica y útil pues, a su facilidad de uso, se une toda la potencia de \TeX. Estas características hicieron que \LaTeX se extendiese rápidamente entre un amplio sector científico y técnico, hasta el punto de convertirse en uso \textbf{obligado en comunicaciones y congresos, y requerido por determinadas revistas a la hora de entregar artículos académicos}.

}
\section{¿Por qu\'e (no) usar \LaTeX?}{
Existiendo otras alternativas más convencionales para producir documentos, como Word de Microsoft, es natural preguntarse porque debería uno tomarse la molestia de aprender LaTeX.

En la superficie, una de las ventajas de \LaTeX es la calidad profesional de los documentos que puedas generar. Esto es particularmente cierto para documentos que contengan fórmulas o ecuaciones, pero LaTeX tiene muchas aplicaciones más allá de las matemáticas. Documentos de química, física, computación, biología, leyes, literatura, música y cualquier otro tema que se te pueda ocurrir, pueden igual aprovechar la excelente calidad de imprenta de LaTeX.

Una ventaja menos obvia, pero quizá más importante, es que \LaTeX te permite claramente separar el contenido y el formato de tu documento. Como científico, investigador o escritor, esto te da la oportunidad de concentrarte en el <<qué>>, en la parte creativa de tu obra, en generar y escribir ideas. Por su parte el sistema se encargará del <<cómo>> se van a ver esas ideas plasmadas en tu documento. LaTeX, además, realiza de manera automática muchas tareas que de otro modo podrían resultar tediosas: numerar capítulos y figuras, incluir y organizar la bibliografía adecuada, mantener índices y referencias cruzadas.

Finalmente, lo que para algunos es también un punto a favor, todo el software que necesitas para editar, producir, ver e imprimir tus documentos es libre. Esto además quiere decir, para los pragmáticos, que para instalar y usar LaTeX en tu computadora no tienes que gastar ni un sólo centavo.

Sin embargo es también útil conocer los casos en los que buscar otra alternativa pueda ser lo más recomendable. No uses \LaTeX si:

\begin{itemize}
\item No tienes tiempo para aprenderlo. Antes de poder usarlo, tienes que aprender a usarlo, no es recomendable tratar de aprender \LaTeX para escribir un documento que tienes compromiso de entregar en menos de, digamos, 24 horas.

\item Ya tienes el documento escrito. Es común, por ejemplo, que un estudiante quiera convertir su tesis, ya escrita en Word, al ‘formato’ de \LaTeX para que se vea ‘más bonita’. Aunque técnicamente esto es posible, no es recomendable porque no es fácil y los resultados suelen ser de muy mala calidad. \LaTeX es mucho más que un ‘formato’ para guardar documentos, es una herramienta completa cuyas ventajas principales están en el proceso de crear el documento.

\item Lo que te interesa es el diseño de documentos. Si tu interés, más que en el contenido está en el diseño visual de los documentos entonces \LaTeX no es para ti.
\end{itemize}
}

\section{Tipos de Documento}{
Los documentos básicos que podemos crear en \LaTeX son:

\begin{enumerate}
	\item Cartas
	\item Reportes
	\item Artículos (este es uno)
	\item Libros
	\item Curriculums Vitae
	\item Presentaciones
	\item Cualquier otro que se te ocurra
\end{enumerate}
}

\section{Contenido del Taller}{
Solo hay una forma de aprender a usar \LaTeX y no es otra que empezando a redactar tus propios documentos, en este taller escribiré un pequeño ejercicio que realice con la plataforma Arduino y el lenguaje de programaci\'on Python con el cual, al redactarlo, aprenderemos a especificar lo siguiente en \LaTeX

\begin{enumerate}
	\item Fuentes
	\item Interlineado
	\item Incluir im\'agenes
	\item S\'imbolos matem\'aticos
	\item Algoritmos
	\item C\'odigo fuente
	\item Indexaci\'on
	\item Referencias y Bibliograf\'ia
\end{enumerate}
}


\end{document}